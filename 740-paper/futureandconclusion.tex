\section{Future Work and Conclusions}
\label{sec-futureandconclusion}

In this section we discuss some of the possible avenues to extend this work, then present the implications of our study and finally conclude. As a future work, we think to study other CC services and compare different services in terms of traffic characters, usage patterns, user base diversity etc. We also intend to develop a much more generic and sophisticated gray-box framework to study CC services. Our current implementation of the toolbox is highly specific to \textit{Twilio}. We also started looking at the security aspects of the \textit{Twilio} servers by doing port scans using \textit{nmap}. From our initial analysis, the servers have only ports 80 and 443 open. As a future work, we would like to analyse if some impersonation attacks are possible with \textit{AuthTokens}. We also want to study how misbehaving clients can attack or cause some form of interference in the system.

Our study has implications for both application developers and \textit{Twilio} developers. First, we showed different components of the \textit{Twilio} ecosystem and how they interact. Our study gives more insights for developers to build robust applications atop \textit{Twilio} platform. We also showed how \textit{Twilio} meets its guarantees in terms of call and message dequeuing rates. This gives the application developers a good sense of what to expect from the \textit{Twilio} service. Second, we should some possible improvements to \textit{Twilio} in sections \ref{sec-oddities} and section \ref{sec-discussion}.

To conclude, we developed a simple VoIP service atop \textit{Twilio} platform and a graybox toolset to gain insights in to the protocols and the architecture of the system. We also empirically measured call and message queue/dequeue rates. We showed some interesting oddities in the service and the client library. We also pointed out to some possible enhancements to the \textit{Twilio} ecosystem. We showed that our study has implications for both application developers and \textit{Twilio} developers. Our study is a small step towards studying the rapidly growing Cloud Communication Services arena and we strongly believe further research is required in exploring this area. 