\section{Twilio Overview}
\label{sec-twilioecoandprotocolstudy}
In what follows, we describe the Twilio Ecosystem, the high level protocol study and the message level protocol study.
\subsection{Twilio Ecosystem}
The twilio ecosystem, as depicted in the Figure~\ref{fig:ecosystem} can be viewed as a layered architecture. In the bottom most layer sits the $Twilio$  $Servers$. These servers exposes a set of data and voice communication APIs for sending and receiving voice calls and messages. In the middle layer lies the $Application$ $Servers$. These servers are installed by Twilio customers with Twilio accounts. For e.g. These application servers might belong to some company X who want to provide VoIP service to its customer.  

Each Twilio account is linked to a 1) Twilio number 2) Account SID 3) Auth Token. A Twilio number is ten digit phone number. Account SID is unique identifier for a Twilio account. Auth Token is a token used by Twilio Servers to authenticate Application servers. The Application server uses the Account SID and Auth Token to access the Twilio APIs. The cost model for Application servers is typically pay per message or pay per call. On the top most layer are the Clients which could be browser based client or phone based clients. For e.g. These clients could be the customers of company X who is providing VoIP service. These clients can be free customers or paid customers of company X. The Clients in order to use the service and communicate with each other need to register with the Twilio Server. This is done using a capability token which is provided by the Application Server. 
\begin{figure}
\centering
%\begin{minipage}{.45\textwidth}
  \centering
  \includegraphics[width=0.45\textwidth]{figs/Ecosystem.png}
  %Mazu_frame_new.png}
\caption{Twilio Ecosystem}
\label{fig:ecosystem}
\end{figure}     

\emph{Application Container: } Each Twilio number is linked to a application container. The application container contains two URLS: Voice URL and Message URL. These URLs are configurable and are configured by the App Server administrators (Twilio customers). Whenever a incoming call or meesages for a Twilio number arrives, the Twilio server makes a post request on these URLs. The content generated by these URLs direct Twilio Server to perform the needed actions on those incoming calls or messages. These contents are in form of a special markup language called $TwiML$.

\emph{TwiML:  } TwiML is a markup language developed by Twilio. It is a set of instructions in form of verbs that can be used by the Application Servers to tell Twilio what to do when a call or message is received to its number. Various kinds of verbs are supported as shown below which can be used to create interactive applications atop Twilio.
\begin{itemize}
\item  Say - Read text to the caller 
\item Play - Play an audio file for the caller
\item Dial - Add another party to the call
\item Record - Record the caller's voice
\item Gather - Collect digits the caller types on their keypad
\item Sms - Send an SMS message during a phone call
\item Hangup - Hang up the call
\item Queue - Add the caller to a queue of callers.
\item Redirect - Redirect call flow to a different TwiML document.
\item Pause - Wait before executing more instructions
\item Reject - Decline an incoming call without being billed.
\end{itemize}

For e.g. The TwiML snippet shown in Figure~\ref{fig:TwilML} will say Hello World to the caller.
 
\begin{figure}
\centering
%\begin{minipage}{.45\textwidth}
  \includegraphics[width=0.25\textwidth]{figs/TwiML.png}
\caption{Sample TwiML Snippet}
\label{fig:TwilML}
\end{figure} 