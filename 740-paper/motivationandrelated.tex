\section{Motivation and Related Work}
\label{sec-motivationandrelated}
Businesses want to delegate communication from their applications and services because of the advantages provided by CCS. The number of enterprises using CCS has seen a steady increase since its advent. Though there are no enough evidences that this pattern is going to continue, because cloud communication services provide an attractive cost-model and easy-to-use APIs, we strongly believe that this trend is going to continue in the future. As more and more enterprises start using CC services like \textit{Twilio}, there is a good chance that this traffic may contribute to a good fraction of Internet traffic in the future. \par
There have been lot of studies on VoIP services like \textit{Skype} in the past [1,2]. There have been recent studies on cloud storage services like \textit{Dropbox} [3]. Best to our knowledge, we are the first to study cloud communication services. Similar studies have been carried out previously - [3] provides a thorough characterization of the Dropbox protocol and traffic patterns. They provide insights into how traffic to \textit{Dropbox} varies across four different networks including home and campus networks. Our study on the other hand does not deal with traffic analysis since we did not have the sophistication of collecting the packet traces on the campus network. Our study aims at studying the architecture and protocols of the entire \textit{Twilio} ecosystem. Our study also aims at finding some possible enhancements to the entire \textit{Twilio} ecosystem. [1] provides a detailed packet level study of the \textit{Skype} protocol. The authors also provide deep insights into the \textit{Skype} architecture. Our study involves studying the architecture of the system and the protocols involved using a suite of gray box tools that we have developed. \par
The reminder of the paper is organized as follows. In section 4, we provide a detailed analysis of the \textit{Twilio} ecosystem, architecture of the system, high level protocols for some key scenarios and packet level analysis for some scenarios. We give detailed measurements with respect to call and message dequeuing rates in section 5. Then, we present some oddities that we found in the entire ecosystem in section 6. In section 7, we discuss some of the possible enhancements to the system. Finally we conclude by presenting the implications of our study and our future work.