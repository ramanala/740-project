\section{Discussion}
\label{sec-discussion}
We now discuss some possible enhancements to the entire \textit{Twilio} ecosystem. 

\emph{\textbf{Client Library Improvements:}}
We showed that the client library can be optimized to save few RTTs in the previous section. It is important to notice that we identified this just by manually studying the behavior of the client library. We believe that there should be more robust and extensive ways to study the client's interaction with the server. We also showed in the previous section that fault handling mechanisms are not uniform across the system spanning the service and the client library. Also, in most of the cases that we tested by inducing faults from the caller side, we found that the callee is notified of the error conditions instead of caller. We believe that, in such scenarios it is best to inform caller with consistent error notifications since it is the caller who can take actions against those error and correct the errors.


\emph{\textbf{Security:}} We found that presently there is no way for Application servers to authenticate the Twilio servers when the later makes the post requests. In the interest of securing the application servers from attacks, we believe Twilio may provide IP whitelists which can be used by the Application servers to authenticate requests. Even IP whitelists might not be sufficient for scenarios in which Application servers are located behind proxies and the IP from which request originates gets masked (e.g. application server hosted behind load balancer in PaaS architecture). For such scenarios some more robust security measures needs to be provided.

\emph{\textbf{Intelligent TwiML fetching:}} As we have shown in Section~\ref{subsec-protostudy}, in case of automated calls to phone, Twilio server fetches the TwiML only after placing the call. We understand that this technique saves extra RTTs in case the call is rejected by the phone. But from the perspective of a callee and assuming that most calls will be answered, we believe it is best to prefetch TwiML before placing the call. Also, we see that in the current automated call model the application server is contacted each time a call is placed. One of the most popular use of automated call is to send/broadcast a single message to multiple numbers (e.g. to send promotional messages or alerts). In such scenarios, it is wasteful to flood the application server to fetch the same content multiple times both from Application server as well as Twilio server point of view. To handle such scenario Twilio might provide some special options using which the application server can indicate that the same message has to used for a specified number of calls or for some specified period of time and the Twilio server can then fetch the request once and cache it.   
 