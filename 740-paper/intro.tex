\section{Introduction}
\label{sec-intro}
CCS is an upcoming service model which provides sophisticated APIs for enterprises to develop applications and offload communication related tasks from enterprise applications. 
%CC services host switching, storage and telephony infrastructure in the cloud. 
All the communication services are offered through simple REST APIs for the applications to make use of them. Cloud communication services have a lot of advantages compared to \textit{on-premise hosted} communication infrastructure. Firstly, it offloads the burden of communication from the applications and separates it as a separate service. The applications can seamlessly interact with the communication APIs to accomplish complex communication tasks like sending promotional messages to customers, providing critical real-time information to mobile phones, etc. Secondly, the development effort involved in integrating an application with CCS is very less compared to developing and maintaining a home-brewed communication infrastructure.
Thirdly, enterprises can build communication applications in a cost effective way because of the pay-per-use model provided by most the CC services. 


Our work focuses on study of such cloud communication services. There are a lot of players in the market which provides CC services like Avaya, Clickatell, Plivo, etc. and we chose a popular CCS \textit{Twilio} for our study purpose. \textit{Twilio} is one of the prominent players in the CCS space and has a huge customer base which includes popular enterprises like \textit{Coca-Cola}, \textit{WalmartLabs}, \textit{Intuit}, \textit{Box}. These enterprises use \textit{Twilio} to accomplish a wide variety of tasks ranging from two-step authentication, powering lending machines with music, secure file sharing, delivering deals and ads to mobile phones, etc. \textit{Twilio} enables lot of new scenarios for businesses that were rather cumbersome to implement in the past. 
The APIs are simple and intuitive to understand and develop. \textit{Twilio} also provides rich documentation and code examples to build simple applications like VoIP communication.
%The entire development and testing took just two developer-weeks. 

 For our study, we developed a simple VoIP service atop Twilio APIs which we call as VoT (VoIP on Twilio). Coupled with logs collected at several places in the system we present a detail study of Twilio which we believe is somewhat representative of CC service in general and is first of its kind. We give insights into the Twilio ecosystem, the high level as well as packet level protocol details and measurement of some guarantees that Twilio provides. We also discuss some interesting oddities that we found during the course of our study and point to some possible enhancements in the system.
 
 
 The reminder of the paper is organized as follows. In section 3, we motivate our study. In section 4, we provide a detailed analysis of the \textit{Twilio} ecosystem, architecture of the system, our experimental setup, some of the scenarios that Twilio supports, high level protocols and packet level analysis for some key scenarios. We give detailed measurements with respect to call and message dequeuing rates in section 5. Then, we present some oddities that we found in the ecosystem in section 6. In section 7, we discuss some of the possible enhancements to the system. Finally we conclude by presenting the implications of our study and our future work. 