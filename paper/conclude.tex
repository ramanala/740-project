\section{Conclusion}
\label{sec-conclude}

\if 0
We present \toolname, a framework that helps understand application update
protocols and detect crash vulnerabilities. We use \toolname\ to analyze 11
applications, finding crash vulnerabilities in all tested applications. Some
vulnerabilities lead to severe consequences like corruption or data loss. We
present several insights from our study of the 11 applications and their
vulnerabilities, which we feel would be useful to application and file-system
developers.   

% Whats the big point we want to make?
Application crash vulnerabilities are like concurrency bugs: they do not occur
in the common case, but when they do occur, they are extremely hard to debug.
In recent years, a great deal of effort has gone into building infrastructure
to detect and reproduce concurrency bugs. We believe this work is a first step
along similar lines for crash vulnerabilities.   
\fi

%Decades of systems research has focussed on how to make distributed systems,
%databases, and file systems consistent after a crash; however, little effort
%has gone into investigating whether this allows the easy development of
%crash-consistent applications, which is what users care about.

In this paper, we show how application-level consistency is dangerously
dependent upon how file systems persist system calls. We show that persistence
properties (file-system behavior affecting application correctness) vary widely
among file systems. We build \toolname, a framework that analyzes
application-level protocols and detects crash vulnerabilities. We use
\toolname\ to analyze 12 applications, finding \totbugs\ vulnerabilities across tested
applications, some of which result in severe consequences like corruption or data
loss. We present several insights derived from our study of application update
protocols and vulnerabilities. We make both our tool and our data publicly
available to stimulate further research in application-level consistency.

