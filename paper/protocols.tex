\section{Update Protocols}
\label{sec-protocols}

In this section, we explain why update protocols vary across
applications. Update protocols have to satisfy two goals: performance and
correctness.  Unfortunately, these goals are often in conflict with each
other. 

If an update protocol is completely synchronous, it is simple and easy to get
right. However, this is extremely slow. Therefore, applications use
asynchronous operations in their protocols, and force persistence using the
\smalltt{fysnc()} system call. However, \smalltt{fsync()} is very expensive,
causing the application thread to block for an arbitrary amount of time
ranging from a few milliseconds to several seconds. Hence, applications use
\smalltt{fsync()} very sparingly in their update protocols. This leads to
complex update protocols.

<<<<<<< HEAD
\begin{figure}[!t]
\centering
    \begin{subfigure}
        \begin{Verbatim}[fontsize=\relsize{-1}]
        lock(application_lock);
        write(user.db);
        unlock(application_lock);
        \end{Verbatim}
    \end{subfigure}
    \vspace{-0.08in}
    {\footnotesize (a)}
    \vspace{-0.08in}
    \begin{subfigure}
        \begin{Verbatim}[fontsize=\relsize{-1}]
=======
\begin{lstlisting}[float=t, caption = {\textbf{Application Update Protocols. }\textit{\footnotesize The
listing shows pseudo-code for three different update protocols for sqlite. the
first protocol is vulnerable to either a process or system crash. the second
protocol is correct in the face of a process crash, but vulnerable in the event
of a system crash. the last protocol is correct in the face of both process and
system crashes.}}, label = {fig-protocols}, escapechar=!]
        lock(application_lock);
        write(user.db);
        unlock(application_lock);
		!\vspace{-0.1in}!
		!{{\normalfont{\footnotesize (a)}}\vspace{0.1in}!
>>>>>>> 76fa2bc03e749a67ec4d2c9755219a96eda9be9d
        lock(application_lock);
        write(journal);
        write(user.db);
        unlock(application_lock);
<<<<<<< HEAD
        \end{Verbatim}
    \end{subfigure}
    \vspace{-0.08in}
    {\footnotesize (b)}
    \vspace{-0.08in}
    \begin{subfigure}
        \begin{Verbatim}[fontsize=\relsize{-1}]
=======
		!\vspace{-0.1in}!
		!{{\normalfont{\footnotesize (b)}}\vspace{0.1in}!
>>>>>>> 76fa2bc03e749a67ec4d2c9755219a96eda9be9d
        lock(application_lock);
        write(journal);
        fsync(journal);
        write(user.db);
        fsync(user.db);
        unlock(application_lock);
<<<<<<< HEAD
        \end{Verbatim}
    \end{subfigure}
    \vspace{-0.08in}
    {\footnotesize (c)}
    \vspace{-0.08in}
\mycaption{fig-protocols}{Application Update Protocols}{\footnotesize The
figure shows pseudo-code for three different update protocols for SQLite. The
first protocol is vulnerable to either a process or system crash. The second
protocol is correct in the face of a process crash, but vulnerable in the event
of a system crash. The last protocol is correct in the face of both process and
system crashes.}
\vspace{-0.1in}
\end{figure}


=======
		!\vspace{-0.1in}!
		!{{\normalfont{\footnotesize (c)}}\vspace{0.1in}!
\end{lstlisting}
>>>>>>> 76fa2bc03e749a67ec4d2c9755219a96eda9be9d


To understand how update protocols can vary in terms of performance and
correctness, we examine the three update protocols shown in
Figure~\ref{fig-protocols}. These are simplified versions of update protocols
used by the embedded database SQLite in different configurations.

Protocol $(a)$ updates the database file using a single write. This protocol is
vulnerable to either a process or system crash: if the \smalltt{write()} call
is interrupted, the database could be partially updated (e.g., only the first
block) leading to corruption.   

Protocol $(b)$ first writes to an application-level journal, and then to the
actual database. If a process crash interrupts the journal write, the database
in unaffected; if the process crashes in the middle of database write, SQLite
can use the journal to restore the database to a consistent state. However,
the protocol is not correct in the event of an inopportune system crash. Upon
completion of the \smalltt{write()} call, the data is stored in operating
system buffers, and can be written to disk in any order. Consider that the
operating system chooses to write out the database data first, and the system
crashes in the middle of the write. Since the database is partially updated, it
is in an inconsistent state. Furthermore, since the journal data was not
written to disk, the journal cannot be used to restore the database to a
consistent state.

Protocol $(c)$ solves this problem by using \smalltt{fsync()} to ensure that
the journal is persisted \textit{before} writing to the database file. A system
crash can only lead to one of two states. In the first case the journal was not
completely persisted.  In this case the update failed and the database remains
in the old, consistent state. In the second case the journal was completely
persisted, but the database may not have been completely updated.  The
journal is read, and the database updated to the new consistent state.
In either case, the database remains consistent. Thus the protocol is not
vulnerable to either process or system crashes.  

As expected, there is a price for safety afforded by crash-consistent
protocols. Protocol $(a)$ has the best performance among the three protocols;
however, it is vulnerable to process and system crashes. Protocol $(c)$ is not
vulnerable to crashes, but it is the slowest among the three protocols. This
performance cost drives developers to use \smalltt{fsync()} sparingly, resulting
in protocols that may perform well but are unsafe.
