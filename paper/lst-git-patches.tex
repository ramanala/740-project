\begin{figure}[!t]
\vspace{-0.2in}
{\scriptsize 
\begin{alltt}
lockfile.c:
int commit_lock_file(struct lock_file *lk) \{
    ...
    result_file[i] = 0;
\textit{    // Fix for dependency between (b) 2-3, (c) 2-5}
\textbf{+   fsync(lk->filename);} 
    ...
\textit{    // Fix for dependency between (b) 3-4, (c) 6-7}
\textbf{+   fsync(parent(lk->filename));}
    lk->filename[0] = 0;
    ...
\}

sha1_file.c:
static int create_tmpfile(char *buffer, ...) \{
    ...
\textit{    // Fix for dependency between (a) 0 and (b) 3}
\textbf{+   fsync(parent(buffer));} 
    /* Try again */
    strcpy(buffer + dirlen - 1, "/tmp_obj_XXXXXX");
    ...    
\}

int move_temp_to_file(const char *tmpfile, ..) \{
    ...
out:
    if (adjust_shared_perm(filename))
        return error(...)
\textit{    // Fix for dependency between (a) 2 and (b) 3}
\textbf{+   fsync(parent(filename));} 
    ...
\}
\end{alltt}
}
\vspace{-0.2in}
\begin{spacing}{0.90}
\mycaption{fig-git-patch}{Patching Git Vulnerabilities}{\footnotesize The figure shows the
lines of code that need to be added to Git to remove the vulnerabilities to
\smalltt{git add} shown in Figure~\ref{fig-git}. We observe that very few lines
of code are necessary; small patches like these are possible because
\toolname\ pinpoints the system calls issued and source lines involved.
}
\end{spacing}
\vspace{-0.2in}
\end{figure}
