\section{Evaluation}
We evaluated \toolname\ on a total of 12 widely used applications. The
applications are representative of different domains. The applications vary in
their level of maturity, ranging from decade-old applications to applications a
few years old. We tested 2 version control systems, 5 key-value stores, 3
relational databases, a virtualization software, and a distributed file system.
For applications that had configuration options which changed the update
protocol significantly, we tested each configuration separately. We tested
different versions of LevelDB since the update protocol changed significantly
between them. In this manner, we test a total of 35 configurations across
applications. 

All applications were tested on a Intel Core i7 quad-core machine with 8 GB of
RAM running Linux kernel 3.8. Crash states were constructed and tested
entirely in-memory. We tested one application at a time, using up-to 8 threads
for testing crash states. 

This section answers the following questions:
\begin{enumerate}
\item Is \toolname\ useful for finding application crash vulnerabilities?
(\sref{sec-report})
\item How long does \toolname\ take to check applications?
(\sref{sec-runningtime})
\end{enumerate}

\subsection{Vulnerabilities Found}
\label{sec-report}
\begin{table}[!t]
\begin{center}
\setlength{\tmpa}{\tabcolsep}
\setlength{\tabcolsep}{2pt}
{\footnotesize
\begin{tabular}{c|c|c|c|c}
% Header
\textbf{Application} & \textbf{Domain} & \textbf{\#VC} & \textbf{\#VS} &
\textbf{\#VL} \\
\hline

% Data
Git & VC  & 1463 & 14 & 7  \\
\hline
Mercurial & VC  & 20840 & 40 & *  \\
\hline
BerkeleyDB & KV & 1024 & 46 & 9  \\
\hline
LevelDB-1.10 & KV & 15487 & 15* & 4* \\
\hline
LevelDB-1.15 & KV & 16963 & 31 & 4  \\
\hline
LMDB  & KV & 1  & 1 & 1 \\
\hline
GDBM  & KV & 704 & 6 & 6  \\
\hline
HSqlDB  & RDB & 1063 & 14 & 9  \\
\hline
SQLite  & RDB & 1 & 1 & 1  \\
\hline
Postgres  & RDB & 1 & 1 & 1 \\
\hline
VMWare & V  & 270 & 1 & 1 \\
\hline
HDFS & DFS & X \\
\end{tabular}
}
\end{center}
\vspace{-0.2in}
\mycaption{tbl-vul-summary}{Application Vulnerabilities}{\footnotesize
The table lists each application tested using \toolname, along with the type
of the application. The number of source code lines involved in a
vulnerability is indicated with the number of vulnerabilities enclosed in
brackets.  Legend: VC -- Vulnerable Crash States. VS -- Vulnerable System
Calls.  VL -- Vulnerable Source Code Lines. VC -- Version control. KV --
Key-Value Store.  RDB -- Relational RDB.  V -- Virtualization. DFS --
Distributed FS.  
} 
\vspace{-0.1in}
\setlength{\tabcolsep}{\tmpa}
\end{table}
 

We tested 12 applications using \toolname. Most of the applications could be
configured in different ways, optimizing for performance or safety. We tested
different configurations of the 12 applications (35 configurations in total).
Since application invariants varied across configurations (e.g., the async mode
of LevelDB does not guarantee that the put is durable), we only report
vulnerabilities that violated the specific invariants of that configuration. 

\toolname\ found 168 vulnerable system calls across the 12 applications we
tested. \toolname\ found vulnerabilities in every application that we tested,
including applications like Git, Mercurial, and LevelDB that are used by
millions of developers across the world.  Table~\ref{tbl-vul-summary} lists the
applications tested, their domains, and the three representations for the
vulnerabilities that were found. We observe that for most applications, a
fairly small number of vulnerable system calls lead to a large number of
vulnerable crash states. For all the applications, all detected vulnerabilities
can be fixed by modifying code at a few locations. We analyze the
vulnerabilities in more detail in Section~\ref{sec-vul-study}. 

\subsection{Performance}
\label{sec-runningtime}

\begin{table}[!t]
\begin{center}
\setlength{\tmpa}{\tabcolsep}
\setlength{\tabcolsep}{1pt}
{\footnotesize
\begin{tabular}{c|c|c|c|c|c}
% Header
\textbf{Application} & \multirow{2}{11mm}{\centering \textbf{\# System Calls}} &
\multirow{2}{10mm}{\centering \textbf{\centering \# Sync Calls}} &
\multirow{2}{11mm}{\centering \textbf{\centering \# Crash States}} &
\multirow{2}{11mm}{\centering \textbf{\centering Checker Latency (s)}} &
\multirow{2}{11mm}{\centering \textbf{Testing\\Time (hours)}} \\
& & & & & \\
& & & & & \\
\hline

SQLite-Rollback & 16 & 4 & 838 & 0.04 & 0.01 \\ \hline 
LMDB & 5 & 2 & 126 & 1.0 & 0.1 \\ \hline 
SQLite-WAL & 42 & 5 & 7934 & 0.04 & 0.1 \\ \hline 
GDBM & 42 & 7 & 4549 & 0.1 & 0.1 \\ \hline 
VMWare & 8 & 13 & 4211 & 0.3 & 0 \\ \hline 
BDB-Hash & 45 & 9 & 7684 & 1.0 & 2 \\ \hline 
BDB-BTREE & 47 & 9 & 10334 & 1.0 & 3 \\ \hline 
Git & 47 & 4 & 6262 & 2.0 & 3 \\ \hline 
HSqlDB & 46 & 6 & 12737 & 1.5 & 5 \\ \hline 
Mercurial & 60 & 0 & 47017 & 1.0 & 13 \\ \hline 
LevelDB-1.10 & 68 & 11 & 155101 & 0.3 & 13 \\ \hline 
LevelDB-1.15 & 85 & 11 & 76321 & 1.0 & 21 \\ \hline 
Postgres & 100 & 18 & 212909 & 4.0 & 237 \\ 

\end{tabular}
}
\end{center}
\vspace{-0.2in}
\mycaption{tbl-vul-time}{Application Testing Time}{\footnotesize
The table shows the time taken to test the safest configuration of each
application.  Application versions with different update protocols are shown
separately. The time taken to test an application depends on the number of
crash states and the latency of the application checker.  The number of crash
states is roughly correlated with the number of system calls in the application
workload and the number of calls like \smalltt{fsync()}.
} 
\vspace{-0.1in}
\setlength{\tabcolsep}{\tmpa}
\end{table}


Table~\ref{tbl-vul-time} shows the time taken by \toolname\ to test different
applications. The total time taken depends on the number of crash states that
are tested and the time taken to check one crash state. The total time varies
between a few minutes for small applications like LMDB to around 10 days for
large applications like Postgres.

The number of crash states for a given application workload depends upon the
number of file-system related system calls and the number of sync calls in the
workload. Each system call introduces a number of intermediate states. The sync
calls introduce dependencies and hence rule out certain crash states.

The time taken to check one crash state depends on the time taken to start each
application, and run application invariant verifiers. For heavyweight
applications like Postgres, startup time is several seconds, bottlenecking
the whole checking process.

\toolname\ is highly \textbf{parallelizable}: since checking one crash state is
completely independent from other tasks, testing all the crash states for an
application can be done using a cluster of machines. This allows \toolname\ to
scale to checking large applications such as Postgres and HDFS.

% just number of sys calls not enough
% size of the write calls etc is important.

% checker time also signfiicant. 
% HDFS runs in parallel (4 machines).


% table of application workloads, system calls, and time taken.

% How big a trace can \toolname\ handle? \\
% What is the limiting factor here? \\
% fsync \\
% table of system calls and fsyncs would be useful here. \\


