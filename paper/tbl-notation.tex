\begin{table}[!t]
\begin{center}
{\footnotesize
\begin{tabular}{c|l}
\textbf{Symbol} & \textbf{Description} \\
\hline
$X^{a}_{i}$ & \multirow{2}{*}{\parbox{5cm}{$X$ is the $i^{th}$ system call in
program. It operates on file/directory $a$}} \\
 & \\
\hline
$W^{a}_{i}$ & $W$ is an overwrite on file $a$ \\
\hline
$A^{a}_{i}$ & $A$ is an append to file $a$ \\
\hline
${TA}^{a}_{i}$ & \multirow{2}{*}{\parbox{5cm}{$TA$ is an append on a file
opened with \smalltt{O\_TRUNC}. Subset of $A$}} \\
 & \\
\hline
$F^{a}_{i}$ & $F$ is one of $\{W, A\}$ on file $a$ \\
\hline
${Dir}^{a}_{i}$ & $Dir$ is a directory op on directory $a$ \\
\hline
$R^{a}_{i}$ & $R$ is a rename operation on file $a$ \\ 
\hline
$S^{a}_{i}$ & $S$ is a sync operation on file/dir $a$ \\ 
\hline
$|X^{a}_{i}|$ & $X$ is atomic \\
\hline
$|A^{a}_{i}|^{P}$ & Prefix-atomic append to file $a$ \\
\hline
$size(F^{a}_{i})$ & Size of the data involved in $F$. \\
\hline
$P(a)$ & Parent directory of file/dir $a$ \\ 
\hline
$DT(X^{a}_{i})$ & \multirow{2}{*}{\parbox{5cm}{Time interval by which $X$ is guaranteed
to be durable}} \\
 & \\
\hline
$X^{a}_{i} \to Y^{b}_{j}$ & $X$ is made durable before $Y$ \\ 
\hline
$[X_i Y_j] \to Z_k$ & \multirow{2}{*}{\parbox{5cm}{$Y$ follows $X$ in program order; both become
durable before $Z$}}. \\ 
& \\
\end{tabular}
}
\end{center}
\vspace{-0.2in}
\mycaption{tbl-notation}{Notation}{\footnotesize The table explains the notation
used for describing persistence properties. $X$, $Y$, and $Z$ denote generic
system calls. See Section~\ref{sec-definition} for
more details.} 
\vspace{-0.1in}
\end{table}
