\section{Related Work}
\label{sec-related}

\newcommand{\EXPLODE}{\textsc{\fontsize{9.2}{5}\selectfont Explode}}
\newcommand{\STACK}{\textsc{\fontsize{9.2}{5}\selectfont Stack}}

Pillai \etal ~\cite{ThanuEtAl13-appconsistency-hotdep} identify the problem of
application consistency depending on file-system behavior, but they are limited
by manual testing, and cover only 2 applications.

\EXPLODE~\cite{YangEtAl06-Explode} has a similar flavor to our work: the
authors use \textit{in-situ} model checking to find crash vulnerabilities on
different storage stacks. Our work differs from \EXPLODE\ in 4 significant
ways. First, \EXPLODE\ requires the target storage stack to be fully
implemented; \toolname\ only requires a model of the target storage stack, and
can therefore be used to evaluate application-level consistency on top of
proposed storage stacks, while they are still at the design stage. Second,
\EXPLODE\ requires the user to carefully annotate complex file systems using
\textit{choose()} calls; \toolname\ requires the user to only specify a
high-level persistence model.  Third, \EXPLODE\ reconstructs crash states by
tracking I/O as it moves from the application to the storage.  Although it is
possible to use \EXPLODE\ to determine the root cause of a vulnerability, we
believe it would be easier to do so using \toolname\, since \toolname\ checks
for violation of specific persistence properties.  Furthermore, \EXPLODE\ stops
at finding crash vulnerabilities; by helping produce protocol diagrams,
\toolname\ contributes to understanding the protocol itself.

Woodpecker~\cite{cui2013verifying} uses known patterns of source code lines
that cause vulnerabilities to find whether the same pattern occurs in other
applications. Our work is fundamentally different as \toolname\ intends to
infer new patterns. 

Our work is influenced by SQLite's internal
testing tool~\cite{sqlite}. The tool works at an internal wrapper layer
within SQLite, and is thus not helpful for generic testing.

OptFS~\cite{Chidambaram+13-OptFS}, Featherstitch~\cite{Frost+07-GenFSDep}, and
transactional file systems~\cite{GallagherEtAl05-LogSkipPaper,
schmuck1991experience, fast09valor}, discuss new file system interfaces
that will affect vulnerabilities. Our study can help inform the designs of new
interfaces, by providing clear insights into what is missing in today's interfaces.
