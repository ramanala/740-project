\begin{table}[!t]
\begin{center}
\setlength{\tmpa}{\tabcolsep}
\setlength{\tabcolsep}{2pt}
{\footnotesize
\begin{tabular}{c|c|c|c|c}
% Header
\textbf{Application} & \textbf{Domain} & \textbf{\#VC} & \textbf{\#VS} &
\textbf{\#VL} \\
\hline

% Data
Git & VC  & 1463 & 14 & 7  \\
\hline
Mercurial & VC  & 20840 & 40 & *  \\
\hline
BerkeleyDB & KV & 1024 & 46 & 9  \\
\hline
LevelDB-1.10 & KV & 15487 & 15* & 4* \\
\hline
LevelDB-1.15 & KV & 16963 & 31 & 4  \\
\hline
LMDB  & KV & 1  & 1 & 1 \\
\hline
GDBM  & KV & 704 & 6 & 6  \\
\hline
HSqlDB  & RDB & 1063 & 14 & 9  \\
\hline
SQLite  & RDB & 1 & 1 & 1  \\
\hline
Postgres  & RDB & 1 & 1 & 1 \\
\hline
VMWare & V  & 270 & 1 & 1 \\
\hline
HDFS & DFS & X \\
\end{tabular}
}
\end{center}
\vspace{-0.2in}
\mycaption{tbl-vul-summary}{Application Vulnerabilities}{\footnotesize
The table lists each application tested using \toolname, along with the type
of the application. The number of source code lines involved in a
vulnerability is indicated with the number of vulnerabilities enclosed in
brackets.  Legend: VC -- Vulnerable Crash States. VS -- Vulnerable System
Calls.  VL -- Vulnerable Source Code Lines. VC -- Version control. KV --
Key-Value Store.  RDB -- Relational RDB.  V -- Virtualization. DFS --
Distributed FS.  
} 
\vspace{-0.1in}
\setlength{\tabcolsep}{\tmpa}
\end{table}
