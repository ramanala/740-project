% File System Persistence Properties and
% Application-Level Crash Consistency: 
% A Tool-Driven Study

\section{Introduction}
\label{sec-intro}

Cloud communication services are Internet-based voice and data communication service providers where telecommunications applications, switching and storage are hosted by a third-party outside of the organization using them, and they are accessed over the public Internet~\cite{wikicc}. 
Cloud communications providers deliver voice & data communications applications and services, hosting them on servers that the providers own and maintain, giving their customers access to the cloud. Because they only pay for services or applications they use, customers have a more cost-effective, reliable and secure communications environment, without the headaches associated with more conventional PBX system deployment.

Companies can cut costs with cloud communications services without sacrificing features.The success of Google and others as cloud-based providers has demonstrated that a cloud-based platform can be just as effective as a software-based platform, but at a much lower cost. Voice services delivered from the cloud increases the value of hosted telephony, as users can equally well turn to a cloud-based offering instead of relying on a facilities-based service provider for hosted VoIP. This expands their options beyond local or regional carriers.[1]

In the past, businesses have been able to do this for IT services, but not telecom. Cloud communications is attractive because the cloud can now become a platform for voice, data and video. Most hosted services have been built around voice, and are usually referred to as hosted VoIP. The cloud communications environment serves as a platform upon which all these modes can seamlessly work as well as integrate.[1]

There are three trends in enterprise communications pushing users to access the cloud and allowing them to do it from any device they choose, a development traditional IT communications infrastructure was not designed to handle. The first trend is increasingly distributed company operations in branches and home offices, making WANs cumbersome, inefficient and costly. Second, more communications devices need access to enterprise networks – iPhones, printers and VoIP handsets, for example. Third, data centers housing enterprise IT assets and applications are consolidating and are often being located and managed remotely.[5]